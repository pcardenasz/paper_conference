\section{Introduction}
The growing demand for cleaner energy sources in power generation due to global warming and environmental issues makes it necessary to use new tools to support the investment decision-making in these projects. Specifically, if one speaks of wind energy, variables such as wind speed, direction and gradient must be predicted effectively and efficiently to respond to the electrical coordination or identify failures due to undesirable mechanical stresses in the turbines. Historically two ways has been taken to predict the wind behavior. The first is an statistical approach that uses data from several meteorological masts or other instrumentation located in a domain to extrapolate information to a time of interest. This approach present some weakness: (i) dependency of the instrumentation, (ii) the historical databases can't capture actual and local conditions of the atmosphere such as climate change, and (iii) the statistical values don't show the behavior of the continuous terrain but in some arbitrary points. Is because this that for more specific goals a second approach is used: the Numerical Weather Prediction (NWP). The NWP target is to find the future state of the meteorological variables by integration of the partial differential equations system  that models the atmosphere behavior. One of the most relevant property of this system is the presence of multiples scales, i.e. the preponderant forces that governs the air dynamics vary depending of the space-temporal scale to analyze. Is convenient then to separate the spatial dependence associating a characteristic length to the domain of interest, in this way a global scale, synoptic scale, mesoscale and microscale can be defined.
%La predicción numérica del clima tiene por objetivo encontrar el estado futuro de las variables meteorológicas integrando el sistema de ecuaciones diferenciales parciales que modelan el comportamiento de la atmósfera. Dentro de las características mas relevantes del problema de resolver este sistema, está la presencia de múltiples escalas, es decir, las fuerzas preponderantes que controlan el movimiento del aire varían dependiendo de la escala espacio-temporal a analizar. Es conveniente separar entonces la dependencia espacial asociando un largo característico al dominio de interés, de esta manera se puede hablar de una escala global, escala sinóptica, mesoescala o microescala. 
This scales multiplicity introduces a new challenge to overcome: the computational cost of solving a system of equations valid for the entire atmosphere. Consequence of this, a spectrum of numerical models have been created specifically for each spatial scale with their own equations, but the initialization of a small-scale model requires the results of a larger one.
%La multiplicidad de escalas introduce un nuevo desafió: superar el costo computacional que significa resolver un sistema de ecuaciones válido para toda la atmósfera. Como consecuencia de esto se han creado una gama de modelos numéricos diseñados específicamente para cada escala espacial, con sus propias ecuaciones, sin embargo la inicialización de un modelo de escala pequeña requiere los resultados de otro mas grande.
With respect to global models, these have shown to be able to correctly simulate many aspects of the general circulation of the atmosphere \citep{stocker2013climate}, however, for engineering interests, the focus is on the local behavior of the wind, specifically, how it moves within the planetary boundary layer (PBL) which is the part of the atmosphere we inhabit and that is outside the resolution of these models.
The approach being used today for small-scale atmospheric simulation, i.e. solving the structures belonging to the PBL, is through the so-called dynamic downscaling, interpolating the results from a large-scale model to a small-scale model in order to function as a boundary condition and generate a forecast in a finer mesh. This method defines what is understood by multi-scale simulation and this type of simulation is still widely discussed by the scientific community \citep{Arnold2010}. The use of dynamic downscaling proved to be successful at least in the spectrum of global and synoptic scales. Numerical issues arise in the mesoscale due to terrain forcing and the relevance of local surface fluxes. The reduction to the microscale causes the increase in the relevance of turbulent stresses in the equations, requiring a much more precise handling. Due to the space-time numerical grid dimensions in large scale models, the turbulence associated with the interaction with the surface and the thermal effects are generally parameterized through a turbulent viscosity model. In scales close to the microscale, the models start natively to solve the turbulent structures generating a problem due to the double weighting of these structures as they are being solved on the one hand and parameterized on the other. This zone is known as the grey zone \citep{Wyngaard2004} and an incorrect configuration of the dynamic downscaling in this zone can cause non-physical results of the model. At the microscale it is possible to represent the turbulence according to known numerical models such as LES or RANS depending on the case. Atmospheric models such as WRF have been widely used in recent years to predict wind behavior at the mesoscale through multiscale simulations, but only a few studies have addressed this behavior at the microscale in real simulations. The researches that analyze the behavior of the LES to represent the PBL generally use ideal conditions (e.g. periodic boundary conditions, flat terrain, imposed pressure gradients) which allows the validation of the approach, but at the cost of losing the operativeness of working in a realistic scenarios. Simulating a real case implies the use of high resolution databases for terrain elevation and land use category.
In this work, in order to obtain the best possible solution for a short-term wind forecast in PBL, the use of a 4D data assimilation system was also considered with measurements obtained in the surface proximity within the simulation time window.

Lastly, the philosophy of this work is to establish the foundations of a new method for assessing the wind resource without relying on ad hoc idealizations, but through fundamental physics and the correct implementation of state-of-the-art instrumentation. The results obtained will serve as a benchmark for future verification of new or experimental models that perform high-resolution simulations in real terrain.