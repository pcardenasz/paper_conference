\bce
%\large{\textbf{Descripción de un Sistema Operacional para Predicción de Viento en Terreno Real a Alta Resolución Mediante Simulación Multiescala, LES y Asimilación de Datos 4D. Parte I: Terreno Plano}}\\
\large{\textbf{Influence of 4D Data Assimilation inside the PBL for High Resolution Multiscale Simulations using WRF-LES in Real Complex Terrain}}\\
%\large{\textit{Description of a Operational Wind Prediction System in High Resolution Real Terrain Through Multiscale Simulation, LES and 4D Data Assimilation, Part I: Flat Terrain}}
%\large{\textit{Influence of 4D Data Assimilation inside the PBL for High Resolution Multiscale Real Simulations using WRF-LES  Part I: Flat Terrain}}
\ece

Pablo Cárdenas$^1$\\
Alex Flores$^{2*}$

\bigskip
$^1$Universidad Técnica Federico Santa María. Departamento de Ing. Mecánica. Valparaíso, Chile.\\
Email: \href{mailto:pablo.cardenasz@alumnos.usm.cl}{pablo.cardenasz@alumnos.usm.cl}\\
$^2$Universidad Técnica Federico Santa María. Departamento de Ing. Mecánica. Valparaíso, Chile.\\
Email: \href{mailto:alex.floresm@usm.cl}{alex.floresm@usm.cl}

$^*$Corresponding Author

\bigskip

\textbf{Abstract:}
In order to develop a reliable tool to estimate the wind behavior in localized areas, a novel methodology that considers fields measurements at surface level was tested for multi-scale atmospherics simulations.
%
Specifically, the WRF software was used to perform real simulations at very high resolution, i.e. mesh sizes up to 2m, through nested domains and the incorporation of non-native databases for orography and land use category. 
%
In the finer domains, the PBL parameterization was turned off and a 1.5TKE LES turbulence model was used. 
%
Finally, measured data was incorporated in the innermost domain through a 4D data assimilation system to correct the numerical deviations every 10 minutes in the first 6 hours of simulation. 
%
Five experiments are presented in two separate scenarios: (I) Høvsøre (quasi-flat terrain with neutral stratification) in which the DA was accomplished through a single-point mast measurement located at the domain center and (II) Bolund (complex terrain with neutral stratification) where the DA was made from 8 masts distributed across the domain.
%measurements of a meteorological mast
%
%This approach was validated by contrasting the obtained results from the flat terrain case with measurement campaigns data and with other data present in the literature. 
Validation of this approach was made with the Høvsøre case by contrasting the obtained results with data from measurements campaigns and literature.
%
The obtained results shows that it is possible to obtain more accurate predictions
that replicate the turbulent wind behavior at simulated scales and that, in addition,
data assimilation improves the flat terrain prediction by 10\%. In complex terrain,
the data assimilation fails to improve the solution due to the proximity of the
measurements with the ground and the terrain induced forcing.\\
\textbf{Keywords:} Multiscale Simulation, LES, Data Assimilation, High Resolution, Complex Terrain, WRF, Planetary Boundary Layer, Wind Resource Assessment.

\bigskip
Received xx de xx de 2019\\
Accepted xx de xx de 2019 