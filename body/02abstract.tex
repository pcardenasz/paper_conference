\bce
%\large{\textbf{Descripción de un Sistema Operacional para Predicción de Viento en Terreno Real a Alta Resolución Mediante Simulación Multiescala, LES y Asimilación de Datos 4D. Parte I: Terreno Plano}}\\
\large{\textbf{High resolution wind resource assessment over complex terrain: influence of 4D data assimilation in the WRF-LES wind model}}\\
%\large{\textit{Description of a Operational Wind Prediction System in High Resolution Real Terrain Through Multiscale Simulation, LES and 4D Data Assimilation, Part I: Flat Terrain}}
%\large{\textit{Influence of 4D Data Assimilation inside the PBL for High Resolution Multiscale Real Simulations using WRF-LES  Part I: Flat Terrain}}
\ece

Pablo Cárdenas$^1,$ Alex Flores$^{1*}$, Joaquín Mura$^{1}$

\bigskip
$^1$Federico Santa María Technical University, Department of Mechanical Engineering, Valparaíso, Chile \\
$^*$Corresponding author email: \href{mailto:alex.floresm@usm.cl}{alex.floresm@usm.cl}

\bigskip

\textbf{Abstract:}
To develop a reliable tool for accurate evaluations of the wind behavior over both flat and complex terrain, a novel methodology that assimilates field measurements at surface level for multiscale wind simulations was implemented and tested in the Weather Research and Forecasting model (WRF-ARW). The proposed method was applied to perform high resolution simulations, i.e. mesh sizes up to 2 m, of real case studies with nested domains and the introduction of a variational data assimilation (DA) technique of field observations, from databases registered for specific experiments over heterogeneous topography. In the inner domains, the atmospheric boundary layer (ABL) parameterization was executed with a 1.5 TKE turbulence scheme for large-eddy simulation (LES), employed to obtain detailed realizations of the anisotropic turbulence of the surface layer. 
Field data was introduced in the innermost domain through a 4D data assimilation system to correct the numerical deviations every 10 minutes in the first 6 hours of the simulation. Four experiments are presented, in two separate scenarios: (i) the Høvsøre case, at Denmark, to replicate a neutrally stratified wind flow over real flat terrain, in which the DA process was accomplished through a three layer single-point met-mast measurement located at the domain center and (ii) the Bolund case, at Denmark, to realize a neutrally stratified wind flow over steep complex terrain, where the DA was taken from 8 met-masts distributed in a neural network across the domain. 
Validation of this approach was made for each case by contrasting the modelling results with field observations from campaign databases and related literature. The outcomes show that it is possible to obtain more accurate predictions of surface wind flow, replicating the nonlinear turbulent phenomena at finescale and that employing data assimilation reduces the wind estimations and predictions over real topography by 10\%. In presence of very steep terrain, the 4D assimilation technique needs further improvement due to the proximity of field measurements to the ground and strong terrain induced forcing, which impose a tough constraint on high resolution wind modelling.\\
\textbf{Keywords:} Wind resource assessment; multiscale wind modelling; data assimilation; large-eddy simulation; complex terrain.